\documentclass[10pt,a4paper,landscape]{article}

\usepackage{compactlayout}

\begin{document}


%------------ Four column -----------------
\noindent
\begin{minipage}[t]{0.24\textwidth}
\textcolor{myred}{\textbf{L1-3 决策树}}
熵: \(H(X)=-\sum_x p(x)\log p(x)\)\\
联合熵: \(H(X,Y)=-\sum_{x,y} p(x,y)\log p(x,y)\)\\
条件熵: \(H(Y|X)=-\sum_{x,y} p(x,y)\log p(y|x)\)\\
交叉熵: \(H(p,q)=-\sum_x p(x)\log q(x)\)\\
KL散度: \(D(p\Vert q)=\sum_x p(x)\log\frac{p(x)}{q(x)}\)\\
互信息: \(I(X;Y)=\sum_{x,y} p(x,y)\log\frac{p(x,y)}{p(x)p(y)}\)\\
\(H(X,Y) = H(X)+H(Y|X)=H(Y)+h(X|Y)\)\\
\(I(X; Y) = H(X) - H(X \mid Y) = H(X) + H(Y) - H(X, Y)\)\\
Vene图:$I(X;Y) = H(X)\cap H(Y);H(X;Y) = H(X)\cup H(Y)$\\
集成方法: Bagging\(\rightarrow\)Random Forest;Boosting\(\rightarrow\)GBDT\\
\textcolor{myyellow}{Splitting Criterion}:衡量split feature方法的好坏的函数。常见:Training ErrorRate($\text{Error}(x_j) = \sum_{i=1}^{m} \frac{|D_i|}{|D|} \cdot \text{Error}(D_i)$)

互信息:\(I(Y;X_i)\):

$I(y; x_d) = H(y) - H(y \mid x_d) = H(y) - \sum_{v \in V(x_d)} f_v \cdot H(Y_{x_d = v})$,

\( H(y) \) 为整个标签集合的熵;\( V(x_d) \) 为特征 \( x_d \) 的所有可能取值集合; \( f_v = \frac{|D_v|}{|D|} \) 为取值为 \( v \) 的样本比例;\( Y_{x_d = v} \) 是所有满足 \( x_d = v \) 的样本标签集合;

\textcolor{myyellow}{pruning}:  Evaluate each split using a \textit{validation} dataset by comparing the validation error rate \textbf{with and without} that split;(Greedily) remove the split that most decreases the validation error rate; Stop if no split improves validation error, otherwise repeat

%%%%%%%%%%%%%L4
\textcolor{myred}{\textbf{L4 KNN}}

对于M个特征,N个数据。Naive:Train$\mathcal{O}(1)$,Predict$\mathcal{O}(MN)$;k-d Tree:Train:$\mathcal{O}(MN \log N)$,Predict,$\mathcal{O}(2^M \log N)$

\textcolor{myyellow}{Experimental Design}:当选好超参数后,最终模型需要在train-subset+validation(all-train)上重新训练

%%%%%%%%%%%%%L4
\textcolor{myred}{\textbf{L5 perceptron}}

权重更新:$\boldsymbol{w} \leftarrow \boldsymbol{w} + \eta y_i \boldsymbol{x}_i$;
偏置更新:$b \leftarrow b + \eta y_i$\\
与$w$垂直的面并且对应的b就是decision boundary;\\
最终的 $\mathbf{w}\;=\;\sum_{i=1}^{N} \alpha_i\, y^{(i)}\,\mathbf x^{(i)}$\\
点到面距离\(
\frac{\| \mathbf{w}^T (\mathbf{x}'' - \mathbf{x}') \|}{\|\mathbf{w}\|_2}
= \frac{\| \mathbf{w}^T \mathbf{x}'' + b \|}{\|\mathbf{w}\|_2}
= \frac{y_i(\boldsymbol{w}^T \boldsymbol{x}_i + b)}{\|\boldsymbol{w}\|}
\)

\textcolor{myred}{\textbf{L6 SVM}}

\textcolor{myyellow}{KernelMethods}:\(K(\mathbf{x},\mathbf{z})=\Phi(\mathbf{x})\Phi(\mathbf{z})\);\(\Phi(\mathbf{x})=\)x的所有维度poly组合,同时可能改变维度

$w_t = a_{i_1} x_{i_1} + \cdots + a_{i_k} x_{i_k}$,$w_t \cdot x = a_{i_1} x_{i_1} \cdot x + \cdots + a_{i_k} x_{i_k} \cdot x
\quad \text{replace with} \quad
a_{i_1} K(x_{i_1}, x) + \cdots + a_{i_k} K(x_{i_k}, x)$

\textcolor{myyellow}{Mercer}:合法核函数满足:1.$K(x, z) = K(z, x)$; 2.$\mathbf{a}^\top \mathbf{K} \mathbf{a} \ge 0$(semi-definite半正定)

\textcolor{myyellow}{核函数转化}: 想要$K(x, z) = c_1 K_1(x, z) + c_2 K_2(x, z)$,只需要更改$\phi(x) = \left( \sqrt{c_1}\, \phi_1(x),\ \sqrt{c_2}\, \phi_2(x) \right)$

\textcolor{myyellow}{优化问题}:  
$\max_{w, \alpha} \gamma$,约束: $\|\mathbf{w}\| = 1$ 且 $y_i(\mathbf{x}_i \cdot \mathbf{w} + \alpha) \ge \gamma$

令 $\mathbf{w}' = \frac{\mathbf{w}}{\gamma}$,$\alpha' = \frac{\alpha}{\gamma}$

原问题等价于:
\(
\min_{\mathbf{w}',\alpha'} \|\mathbf{w}'\|^2
\)且\( \quad y_i(\mathbf{x}_i \cdot \mathbf{w}' + \alpha') \ge 1
\)\\
\textcolor{myyellow}{\textbf{SVM Optimizationl}}\\
{Primal (soft‐margin) form}:  
minimize over $w,b,\{\xi_i\}$ the objective $ \tfrac12\|w\|^2 + C\sum_{i=1}^m\xi_i$  
subject to $y_i\,(w^\top x_i + b)\ge1 - \xi_i$ and $\xi_i\ge0\,$.\\
{Lagrangian}: introduce multipliers $\alpha_i\in[0,C]$, $\mu_i\ge0$, and form  
$\mathcal L=\tfrac12\|w\|^2 + C\sum_i\xi_i 
 - \sum_i\alpha_i\bigl[y_i(w^\top x_i+b)-1+\xi_i\bigr]
 - \sum_i\mu_i\,\xi_i\,$.\\
Stationarity w.r.t.\ $w,b,\xi$ gives $w=\sum_i\alpha_i y_i x_i$, $\sum_i\alpha_i y_i=0$, and $\alpha_i+\mu_i=C\,$. Eliminating $w,b,\xi$ yields the dual QP:  
maximize $\sum_{i=1}^m\alpha_i - \tfrac12\sum_{i,j}\alpha_i\alpha_j y_i y_j\,x_i^\top x_j$  
subject to $\sum_{i=1}^m\alpha_i y_i=0$ and $0\le\alpha_i\le C\,$.\\
{Decision function}: $f(x)=\mathrm{sign}\bigl(\sum_{i=1}^m\alpha_i y_i\,x_i^\top x + b\bigr)$, where support vectors satisfy $0<\alpha_i\le C\,$.\\














\end{minipage}
\hfill
\begin{minipage}[t]{0.24\textwidth}

\textcolor{myred}{\textbf{L7 Regression}}\\
\textcolor{myyellow}{KNN}: store all $(x,y)$ pairs; $k=1$ return nearest $y$; $k=2$ return weighted average $y$.\\
\textcolor{myyellow}{Decision Tree Regression}: model as a binary tree with each internal node testing $x_j\le s$ and each leaf predicting a constant $\hat y_\ell$.\\
Split criterion: choose feature $j$ and threshold $s$ to minimize $\Delta_{\rm MSE}(j,s)=\frac{|\mathcal D_{\rm left}|}{|\mathcal D|}\,\mathrm{MSE}(\mathcal D_{\rm left})+\frac{|\mathcal D_{\rm right}|}{|\mathcal D|}\,\mathrm{MSE}(\mathcal D_{\rm right})$, where $\mathrm{MSE}(\mathcal D)=\frac1{|\mathcal D|}\sum_{(x,y)\in\mathcal D}(y-\bar y)^2$.\\
Leaf prediction: in leaf $\ell$ with data $\mathcal D_\ell$, set $\hat y_\ell=\frac1{|\mathcal D_\ell|}\sum_{(x,y)\in\mathcal D_\ell}y$.\\
    
\textcolor{myyellow}{\textbf{Optimization Method \#1: Gradient Descent}}\\
Residual: $e^{(i)}=y^{(i)}-\hat y^{(i)}$.\\
MSE objective: $J(\theta)=\frac1N\sum_{i=1}^N\bigl(y^{(i)}-(\theta^T x^{(i)}+b)\bigr)^2$.\\
Gradient: $\nabla_\theta J(\theta)=\begin{bmatrix}
\frac{\partial J}{\partial \theta_1} & \cdots & \frac{\partial J}{\partial \theta_M}
\end{bmatrix}^T
=\sum_{i=1}^N\bigl(\theta^T x^{(i)}-y^{(i)}\bigr)x^{(i)}$.\\
Gradient descent update: $\theta\leftarrow\theta-\gamma\,\nabla_\theta J(\theta)$.\\
Algorithm: initialize $\theta^{(0)}$; while not converged do $g=\sum_{i=1}^N(\theta^T x^{(i)}-y^{(i)})x^{(i)}$, $\theta\leftarrow\theta-\gamma g$; end; return $\theta$.\\
Test time: $\hat y=h_\theta(x)=\theta^T x\,$.\\
\textcolor{myyellow}{\textbf{Closed‐form Solution for Linear Regression}}\\
Minimize MSE: $J(\theta)=\frac1{2N}\sum_{i=1}^N\bigl(y^{(i)}-\theta^T x^{(i)}\bigr)^2 
= \frac1{2N}(X\theta - y)^T(X\theta - y) 
= \frac1{2N}\bigl(\theta^T X^T X\theta -2\theta^T X^T y + y^T y\bigr)$.\\
Set gradient to zero: $\nabla_\theta J(\theta)=\frac1{2N}(2X^T X\theta -2X^T y)=0 
\;\Longrightarrow\;X^T X\hat\theta = X^T y$.
Closed‐form solution: $\hat\theta=(X^T X)^{-1}X^T y\,$.\\
Uniqueness:只要你的特征之间线性相关,collinearity,模型参数解就可能不唯一,存在无穷多组最优解。\\
Core formula: $\hat\theta=(X^T X)^{-1}X^T y$, valid iff $X^T X$ invertible.\\
Q1: Invertibility holds when $\mathrm{rank}(X)=D+1$, e.g.\ $N\gg D+1$ and no feature collinearity (if e.g.\ $x_3=2x_1+5x_2$, then not).\\
Complexity: compute $X^T X\in\mathbb R^{(D+1)\times(D+1)}$ in $O(ND^2)$; invert it in $O(D^3)$; total $O(ND^2+D^3)$.\\
closed‐form is fast  unique but requires full rank; otherwise use GD/SGD or regularization\\
\textcolor{myyellow}{SGD}: Sample index $i\sim\mathrm{Uniform}\{1,2,\dots,N\}$. Compute gradient $g=\nabla_\theta J^{(i)}(\theta)$(单样本). Update $\theta\leftarrow\theta-\gamma\,g$.


Derivative of per‐example loss $J^{(i)}(\theta)=\tfrac12(\theta^T x^{(i)}-y^{(i)})^2$:\\  
$\frac{\partial}{\partial\theta_k}J^{(i)}(\theta)
=\frac{\partial}{\partial\theta_k}\,\frac12\bigl(\theta^T x^{(i)}-y^{(i)}\bigr)^2
=\bigl(\theta^T x^{(i)}-y^{(i)}\bigr)\,\frac{\partial}{\partial\theta_k}\bigl(\theta^T x^{(i)}-y^{(i)}\bigr)
=\bigl(\theta^T x^{(i)}-y^{(i)}\bigr)\,x_k^{(i)}$\\
Gradient for example $i$:  
$\nabla_\theta J^{(i)}(\theta)
=(\theta^T x^{(i)}-y^{(i)})\,x^{(i)}.$\\
Full‐batch gradient of $J(\theta)=\frac1N\sum_i J^{(i)}(\theta)$:  
$\nabla_\theta J(\theta)
=\frac1N\sum_{i=1}^N\nabla_\theta J^{(i)}(\theta)
=\frac1N\sum_{i=1}^N(\theta^T x^{(i)}-y^{(i)})\,x^{(i)}.$

\textcolor{myyellow}{\textbf{Why SGD Works: Unbiased Gradient Estimation}}\\
Let $i$ be sampled uniformly from $\{1,\dots,N\}$, so $P(i)=1/N$.\\
ERM objective: $J(\theta)=\frac1N\sum_{i=1}^N J^{(i)}(\theta)$.\\
Expectation definition: $E_i[f(i)]=\sum_{i=1}^N P(i)\,f(i)$.\\
SGD gradient expectation:\\
$E_i[\nabla_\theta J^{(i)}(\theta)] 
=\sum_{i=1}^N\frac1N\,\nabla_\theta J^{(i)}(\theta) 
=\nabla_\theta J(\theta)$.\\
$\Rightarrow$ single‐sample gradient is an unbiased estimator of the full gradient.\\
Intuition: noisy but in expectation follows the true direction.\\


















\end{minipage}
\hfill
\begin{minipage}[t]{0.24\textwidth}

\textcolor{myred}{\textbf{L12-13 Neural Network}}\\
\textcolor{myyellow}{Perceptron}: \(a=\boldsymbol w^T\boldsymbol x+b,\;z=\phi(a)\).\\
\textcolor{myyellow}{MLP Forward}: for layer \(l\): \(a^{(l)}=W^{(l)}z^{(l-1)}+b^{(l)},\;z^{(l)}=\phi(a^{(l)})\). Final output \(\hat y=z^{(L)}\). Loss: \(\mathcal L(\hat y,y)\).\\
\textcolor{myyellow}{\(\delta\)误差项}: define \(\delta^{[l]}= \frac{\partial \mathcal{L}}{\partial z^{[l]}}\frac{\partial z^{[l]}}{\partial a^{[l]}}=\frac{\partial\mathcal L}{\partial z^{[l]}}\circ\phi'(a^{[l]})\). \\
For output layer \(L\): \(\delta^{[L]}=\frac{\partial\mathcal L}{\partial\hat y}\circ\phi'(a^{[L]})\)\\
\(\delta^{[l-1]}=(W^{[l]})^T\delta^{[l]}\circ\phi'(a^{[l-1]})\).\\
\textcolor{myyellow}{Gradients}: \(\frac{\partial\mathcal L}{\partial W^{[l]}}=\delta^{[l]}\,(z^{[l-1]})^T,\;\frac{\partial\mathcal L}{\partial b^{[l]}}=\delta^{[l]}\).\\
\textcolor{myyellow}{Update}: \(W^{[l]}\!:=W^{[l]}-\eta\,\frac{\partial\mathcal L}{\partial W^{[l]}},\;b^{[l]}\!:=b^{[l]}-\eta\,\frac{\partial\mathcal L}{\partial b^{[l]}}\).\\


\textcolor{myred}{\textbf{L11 Feature Engineering \& Regularization}}\\
\textcolor{myyellow}{Learned Embedding}: $f_{\text{deep}}(x)=\text{NN}(x)$.\\
\textcolor{myyellow}{Polynomial Basis}: $\phi_k(x)=x^k,\;k=0,\dots,m$.\\
\textcolor{myyellow}{Kernel Trick}: $K(x,x')=\langle\phi(x),\phi(x')\rangle$.\\
\textcolor{myyellow}{Reg Obj}: $\min_\theta\; \mathcal L(\theta)+\lambda\,r(\theta)$.\\
$\|\theta\|_q \;=\; \left( \sum_{m=1}^{M} |\theta_m|^{\,q} \right)^{\frac{1}{q}}$\\
\textcolor{myyellow}{L2 Ridge}: $r(\theta)=\|\theta\|_2^2\;\Rightarrow\;\theta^*=(X^{\!T}X+\lambda I)^{-1}X^{\!T}y$.\\
\textcolor{myyellow}{L1 Lasso}: $r(\theta)=\|\theta\|_1\;\Rightarrow\;\text{sparse }\theta^*$.不可导\\
\textcolor{myyellow}{Gradient Update}: $\theta:=\theta-\eta\bigl(\nabla\mathcal L+\lambda\nabla r\bigr)$.\\
\textcolor{myyellow}{CV Tune $\lambda$}: pick $\lambda=\arg\min_{\lambda}\mathcal L_{\text{val}}$.\\




\textcolor{myred}{L14-15 CNN}\\
k:卷积核大小,p:填充,s:步长

\(y_{i,j}^{(c)}=\sum_{u=1}^{k}\sum_{v=1}^{k}\sum_{c'=1}^{C_{\text{in}}}
               W_{u,v}^{(c,c')}x_{i+u,j+v}^{(c')}\).\\
\(H_{\text{out}}=\Bigl\lfloor\frac{H_{\text{in}}+2P-k}{S}\Bigr\rfloor+1,\;
                                   W_{\text{out}}=\Bigl\lfloor\frac{W_{\text{in}}+2P-k}{S}\Bigr\rfloor+1\).\\
\textcolor{myyellow}{\#Params}: \(k^{2}C_{\text{in}}C_{\text{out}}+bias(C_{out})\).\\
\textcolor{myyellow}{Receptive Field}: 视野域,由卷积核k决定\\
\textcolor{myyellow}{Pooling}: Strid $S$\\
\textcolor{myyellow}{Equivariance}: \(f(Tx)=T\,f(x)\);\;Invariance via pooling.\\

\textcolor{myyellow}{ReLU}: \(f(a)=\max(0,a)\).\\
\textcolor{myyellow}{Leaky ReLU}: \(f(a)=\max(\alpha a,a)\).\\
\textcolor{myyellow}{ELU}: \(f(a)=\begin{cases}a,&a>0\\ \alpha\!\bigl(e^{a}-1\bigr),&a\le0\end{cases}\).\\
\textcolor{myyellow}{Sigmoid}: \(\sigma(a)=\dfrac{1}{1+e^{-a}}\).\\
\textcolor{myyellow}{tanh}: \(\tanh(a)=\dfrac{e^{a}-e^{-a}}{e^{a}+e^{-a}}\).\\
\textcolor{myyellow}{Maxout}: \(f(a)=\max_{k}\bigl(w_k^{\!\top}x+b_k\bigr)\).\\


\textcolor{myyellow}{\textbf{Model Complexity}}\\
\textcolor{myyellow}{FLOPs (Conv)}: \(2k^{2}C_{\text{in}}C_{\text{out}}H_{\text{out}}W_{\text{out}}\).\\
\textcolor{myyellow}{Same‐Pad Rule}: keep size when \(P=\tfrac{k-1}{2},\,S=1\).\\

\textcolor{myyellow}{\textbf{Representative Architectures}}\\
\textcolor{myyellow}{Residual Block}: \(y = F(x,W)+x\).\\
\textcolor{myyellow}{DenseNet}: \(x_{l}=H_{l}\bigl([x_{0},x_{1},\dots,x_{l-1}]\bigr)\).\\
\textcolor{myyellow}{1\(\times\)1 Conv}: feature mixing per pixel, acts like FC layer.\\

\textcolor{myred}{\textbf{Upsampling \& Dilated Conv}}\\
\textcolor{myyellow}{Transpose Conv (Size)}:
   \(H_{\text{out}}=(H_{\text{in}}-1)S-2P+k_{\text{eff}}\).\\
\textcolor{myyellow}{Dilated kernel}: \(k_{\text{eff}} = k + (k-1)(d-1)\).\\
















\end{minipage}
\hfill
\begin{minipage}[t]{0.24\textwidth}


\textcolor{myred}{L18 RNN}

$\begin{aligned}
h_t &= \tanh(W_{xh} x_t + W_{hh} h_{t-1} + b_h) \quad &\text{(隐藏状态更新)} \\
y_t &= W_{hy} h_t + b_y \quad &\text{(输出)} \\
\end{aligned}$

\textcolor{myyellow}{LSTM}

$\begin{aligned}
i_t &= \sigma\!\bigl(W_{xi}\,x_t + W_{hi}\,h_{t-1} + b_i\bigr)             &\text{(输入门)} \\[2pt]
f_t &= \sigma\!\bigl(W_{xf}\,x_t + W_{hf}\,h_{t-1} + b_f\bigr)             &\text{(遗忘门)} \\[2pt]
o_t &= \sigma\!\bigl(W_{xo}\,x_t + W_{ho}\,h_{t-1} + b_o\bigr)             &\text{(输出门)} \\[2pt]
g_t &= \tanh\!\bigl(W_{xg}\,x_t + W_{hg}\,h_{t-1} + b_g\bigr)              &\text{(候选记忆)} \\[4pt]
c_t &= f_t \odot c_{t-1} + i_t \odot g_t                                   &\text{(记忆单元更新)} \\[4pt]
h_t &= o_t \odot \tanh(c_t)                                                &\text{(隐藏状态)} \\[4pt]
y_t &= W_{hy}\,h_t + b_y                                                   &\text{(输出)} \\
\end{aligned}$


% ==================== Transformer Self-Attention ====================

\textcolor{myred}{L19 Attn}
\(S = \frac{QK^{\!\top}}{{d_k}^{\frac{1}{2}}}\).
\(A=\operatorname{softmax}(S+M)\)\;(mask \(M\!=-\infty\) above diag for causal).\\
\(X' = AV\).
 \(Q = XW_q,\; K = XW_k,\; V = XW_v\).\\
\(X=[x_1,\dots,x_T]^{\top}\in\mathbb R^{T\times d_{\text{model}}}\).\\

\textcolor{myyellow}{{Multi-Head Attn (H heads)}}\\
{Per-Head}: \(Q^{(i)}\!=\!XW^{(i)}_q,\;K^{(i)}\!=\!XW^{(i)}_k,\;V^{(i)}\!=\!XW^{(i)}_v\).\\
{Head Out}: \(X'^{(i)}=\operatorname{softmax}(\dfrac{Q^{(i)}K^{(i)\!\top}}{d_k^{\frac{1}{2}}}+M)V^{(i)}\).\\
{Concat}: \(X=\operatorname{concat}\bigl(X'^{(1)},\dots,X'^{(H)}\bigr)\).\\
% ==================== Pre-Training vs. Fine-Tuning ====================
\textcolor{myyellow}{Pre-Training}\\
{Init}: start from \emph{random} weights.\\
{Mode A (unsup.)}: maximise likelihood / reconstr.\ on huge \underline{unlabeled} set.\\
{Mode B (sup.)}: train on huge labeled set (e.g.\ ImageNet-21k, 14 M imgs).\\
{Vision ex.}: autoencoder on MNIST; ImageNet cls.\ 21 k classes.\\
{Language ex.}: The Pile (800 GB), Dolma (3 T tokens).\\[2pt]
\textcolor{myyellow}{\textbf{Fine-Tuning}}\\
{Init}: load \emph{pretrained} weights.\\
(Opt.) Head: add small randomly-init prediction head.\\
{Train}: back-prop on task-specific dataset.\\
{Vision ex.}: COCO det.\ (200 k imgs), ADE20K seg.\\
{NLP ex.}: MMLU few-shot (57 tasks); MBPP code-gen .\\
















\end{minipage}
%------------ Four column end -------------


\newpage

%------------ Four column -----------------
\noindent
\begin{minipage}[t]{0.24\textwidth}
\textcolor{myyellow}{Recommender Systems \& Collaborative Filtering}:\\
Task: predict unknown user–item ratings in a sparse matrix \(R\!\in\!\mathbb R^{m\times n}\); quality often measured by RMSE.\\
Paradigms: content-based (use side features) vs.\ collaborative (use interaction data only).\\
CF families: neighborhood methods (user- / item-based similarity) and latent-factor methods (low-rank matrix factorization).\\

\textcolor{myyellow}{Matrix Factorization (MF) Core Equations}:\\
Model: \(R \approx UV^{\!\top}\) with \(U\!\in\!\mathbb R^{m\times k},\;V\!\in\!\mathbb R^{n\times k}\).\\
Loss : \(J=\frac12\sum_{(i,j)\in\Omega}(R_{ij}-u_i^{\!\top}v_j)^2+\frac\lambda2(\|U\|_F^2+\|V\|_F^2)\).\\
SGD update: \(u_i \leftarrow u_i+\eta(e_{ij}v_j-\lambda u_i),\;v_j \leftarrow v_j+\eta(e_{ij}u_i-\lambda v_j)\) where \(e_{ij}=R_{ij}-u_i^{\!\top}v_j\).\\
ALS: alternately fix \(V\) and solve \(U\), then fix \(U\) and solve \(V\) via least squares.\\
Variants: SVD for fully observed data, non-negative MF, implicit-feedback MF, bias terms, etc.

\textcolor{myyellow}{Ensemble Learning}:\\Bagging (parallel bootstrap) and Boosting (sequential re-weighting).

\textcolor{myyellow}{Bagging \& Random Forests}:\\
Bootstrap Bagging: draw \(S\) bootstrap samples, train \(T\) base models, aggregate by majority vote(regression compute mean).\\
Feature Bagging: each learner sees only a random subspace of features.\\
Random Forest: bootstrap samples \emph{plus} random feature subset at every tree split.\\(OOB error: ~37 \% out-of-bag )

\textcolor{myyellow}{Weighted Majority Algorithm}:\\
Online setting with \(N\) experts; start equal weights; prediction by weighted vote; if expert errs, multiply its weight by \(\beta\!\in\!(0,1)\).


\textcolor{myyellow}{AdaBoost}:\\
Initial distribution \(D_1(i)=1/N\).\\
At round \(t\): train weak learner \(h_t\) w, error \(\varepsilon_t\); set \(\alpha_t=\tfrac12\ln\frac{1-\varepsilon_t}{\varepsilon_t}\).\\
Update \(D_{t+1}(i)=\dfrac{D_t(i)\exp(-\alpha_t y_i h_t(x_i))}{Z_t}\).\\
Final classifier \(H(x)=\operatorname{sign}\bigl(\sum_{t=1}^{T}\alpha_t h_t(x)\bigr)\).\\
Properties: empirical error drops exponentially; margin theory explains generalization.

\textcolor{myyellow}{Key Comparison}:\\
Bagging — variance reduction, fully parallel, excels with high-variance bases.\\
Boosting — bias reduction, turns weak into strong, sensitive to noisy labels/outliers.\\
Random Forest — efficient, parallelizable, offers OOB validation and feature interpretability.\\






\end{minipage}
\hfill
\begin{minipage}[t]{0.24\textwidth}
\textcolor{myred}{L10 Logistic}

% ------------------- Logistic Regression Cheat-Sheet -------------------
\( f(x)=\sigma(\theta^{\top}x+b) \),\; where  
\( \sigma(z)=\dfrac{1}{1+e^{-z}} \).\\
\( P(y=1\mid x)=\sigma(\theta^{\top}x+b) \),\;

\textcolor{myyellow}{Objective (negative log-likelihood)}:\\
\( \ell(\theta)= -\dfrac{1}{N}\log P(y^{(1)},\dots ,y^{(N)}\mid x^{(1)},\dots ,x^{(N)},\theta) \)\\
\( =-\dfrac{1}{N}\log\prod_{n=1}^{N}P(y^{(n)}\mid x^{(n)},\theta) \)\\
\( =-\dfrac{1}{N}\log\prod_{n=1}^{N}\bigl(P(Y\!=\!1\mid x^{(n)},\theta)\bigr)^{y^{(n)}}\!
                                   \bigl(P(Y\!=\!0\mid x^{(n)},\theta)\bigr)^{1-y^{(n)}} \)\\
\( =-\dfrac{1}{N}\sum_{n=1}^{N}y^{(n)}\log P(Y\!=\!1\mid x^{(n)},\theta)
                             +(1-y^{(n)})\log P(Y\!=\!0\mid x^{(n)},\theta) \)\\
\( =-\dfrac{1}{N}\sum_{n=1}^{N}y^{(n)}\theta^{\top}x^{(n)}
        -\log\!\bigl(1+e^{\theta^{\top}x^{(n)}}\bigr) \).
        
\textcolor{myyellow}{Gradients}:\\
\( J(\theta)=\ell(\theta) \).\\
\( \displaystyle\nabla_{\theta}J(\theta)=
      \frac{1}{N}\sum_{n=1}^{N}x^{(n)}\bigl(\hat y^{(n)}-y^{(n)}\bigr),\;
   \hat y^{(n)}=\sigma(\theta^{\top}x^{(n)}+b) \).\\
Weight update:  
\( \theta:=\theta-\eta\,\nabla_{\theta}J(\theta) \).\\
Bias gradient / update:  
\( \dfrac{\partial J}{\partial b}= \dfrac{1}{N}\sum_{n=1}^{N}(\hat y^{(n)}-y^{(n)}),\;
   b:=b-\eta\,\dfrac{\partial J}{\partial b} \).
   
\textcolor{myyellow}{Bayes Decision Rule}:\\
\( \hat y=\arg\max_{y\in\{0,1\}}P(y\mid x)=
\begin{cases}
1,&\sigma(\theta^{\top}x+b)\ge 0.5\\
0,&\text{otherwise}
\end{cases}
\;\Longleftrightarrow\;
\begin{cases}
1,&\theta^{\top}x+b\ge0\\
0,&\text{otherwise.}
\end{cases} \)\\
Decision boundary: \( \theta^{\top}x+b=0 \).

\textcolor{myyellow}{Why log-likelihood, not likelihood}:\\
1. Turns products into sums → easier calculus.\\
2. \(\log\) is monotone ⇒ optimum unchanged.\\
3. Prevents underflow from tiny products.\\
4. Simplifies gradients, convergence analysis.\\[2pt]

Cross-entropy loss (same expression as \( \ell(\theta) \)) is thus obtained by
maximizing the Bernoulli likelihood of the labels.\\

\end{minipage}
\hfill
\begin{minipage}[t]{0.24\textwidth}


% ============================================ L21 KMeans ====================================
% ============================================ L21 KMeans ====================================
% ============================================ L21 KMeans ====================================


\textcolor{myred}{L21 KMeans}\\
问题定义: N个点,聚类成K个类. 输出: K个center和对于每个点的标签\\
choose the nearest centre: $z^{(i)} \;=\; \arg\min_{j}\, \bigl\|\mathbf{x}^{(i)} - \mathbf{c}_j\bigr\|_2^{2}.$

$
\hat{\mathbf{C}}
    \;=\;
    \arg\min_{\mathbf{C}}
    \sum_{i=1}^{N}
    \min_{j}
    \bigl\|\mathbf{x}^{(i)} - \mathbf{c}_j\bigr\|_2^{2}
    \;=\;
    \arg\min_{\mathbf{C},\,\mathbf{z}}
    \sum_{i=1}^{N}
    \bigl\|\mathbf{x}^{(i)} - \mathbf{c}_{z^{(i)}}\bigr\|_2^{2}
    \;=\;
    \arg\min_{\mathbf{C},\,\mathbf{z}} J(\mathbf{C},\mathbf{z}).
$

$
J(\{C_k\}_{k=1}^{K})
=\sum_{k=1}^{K}\,
\sum_{\mathbf{x}\in C_k}
\bigl\lVert \mathbf{x}-\boldsymbol{\mu}_k \bigr\rVert_2^{2},
\qquad
\boldsymbol{\mu}_k=\frac{1}{|C_k|}\sum_{\mathbf{x}\in C_k}\mathbf{x}.
$\\
Lloyd 算法: 随机初始化,然后repeat until converge:

a. Assignment step: 
$
z^{(i)} \leftarrow \arg\min_{j}
      \bigl\|\mathbf{x}^{(i)} - \mathbf{c}_j\bigr\|_2^{2},
      \quad i=1,\dots,N.
$
b. Update step: 
$
\mathbf{c}_j
      \leftarrow
      \frac{1}{|C_j|}
      \sum_{\mathbf{x}^{(i)}:\,z^{(i)} = j}
      \mathbf{x}^{(i)},
      \quad j=1,\dots,K,
$

FPH:$D(x) = \min_{c \in C} \|x - c\|_2$, Select the $c_k = \arg\max_{x \in X} D(x)$ for next center until have chosen K centers

K-means++:   以$P(\mathbf{x})=\frac{D(\mathbf{x})^2}{\sum_{\mathbf{x}'}D(\mathbf{x}')^2}$ 概率选下一个center

理论误差保证:$\mathbb{E}[\text{Cost}_{\text{K++}}] \leq O(\log K) \cdot \text{OPT}$
















\end{minipage}
\hfill
\begin{minipage}[t]{0.24\textwidth}
% ============================================ L21 EM ====================================
% ============================================ L21 EM ====================================
% ============================================ L21 EM ====================================
\textcolor{myred}{L22 EM+GMM}\\
\textcolor{myyellow}{Parameter Estimation with Latent Variables}:\\

    \( 
      \log p(\mathbf X \mid \Theta)
      = \log \sum_{z} p(\mathbf X,z \mid \Theta)
      = \log \sum_{z} q(z)\,\frac{p(\mathbf X,z \mid \Theta)}{q(z)}
    \)\\
    \( 
      \phantom{\log p(\mathbf X \mid \Theta)}
      \;\ge\;
      \sum_{z} q(z)\,
      \log\frac{p(\mathbf X,z \mid \Theta)}{q(z)}
      \quad
      (\text{concave } f,\ \text{Jensen: } f(\sum \lambda_i x_i) \ge \sum \lambda_i f(x_i))
    \)\\
    \( 
      \log p(\mathbf X \mid \Theta)
      \;\ge\;
      \sum_{z} q(z)\,\log p(\mathbf X,z \mid \Theta)
      -\sum_{z} q(z)\,\log q(z)
      =\sum_{z} q(z)\,\log p(\mathbf X,z \mid \Theta)+\text{const.}
    \qquad
      \bigl(\text{term independent of } \Theta\bigr)
    \)
 If we set \( q(z)=p(z \mid \mathbf X,\Theta) \), the inequality becomes equality\\
    \(
      \sum_{z} q(z)\,
      \log\frac{p(\mathbf X,z \mid \Theta)}{q(z)}
      =
      \sum_{z} p(z \mid \mathbf X,\Theta)\,
      \log\frac{p(z \mid \mathbf X,\Theta)\,p(\mathbf X \mid \Theta)}{p(z \mid \mathbf X,\Theta)}
      =
      \sum_{z} p(z \mid \mathbf X,\Theta)\,\log p(\mathbf X \mid \Theta)
      =
      \log p(\mathbf X \mid \Theta)
    \)
 Thus for \( q(z)=p(z \mid \mathbf X,\Theta) \) we have\\
    \(
      \log p(\mathbf X \mid \Theta)
      =
      \sum_{z} p(z \mid \mathbf X,\Theta)\,\log p(\mathbf X,z \mid \Theta)
      +\text{const.}
      =
      \mathbb E_{p(z \mid \mathbf X,\Theta)}
      \bigl[\log p(\mathbf X,z \mid \Theta)\bigr]+\text{const.}
    \)
 Therefore \( \log p(\mathbf X \mid \Theta) \) is tightly lower-bounded by
        \( \mathbb E\!\bigl[\log p(\mathbf X,z \mid \Theta)\bigr] \),
        which the EM algorithm maximizes.
        
\textcolor{myyellow}{Expectation Maximization (EM) Algorithm}:\\
\textbf{E (Expectation) step}:\\
\quad Compute the posterior \( p(\mathbf Z\!\mid\!\mathbf X,\Theta^{\text{old}}) \) over latent variables \( \mathbf Z \) using \( \Theta^{\text{old}} \).
\quad Compute the expected complete-data log-likelihood with respect to this posterior:\\
\quad \( Q(\Theta,\Theta^{\text{old}})=
        \mathbb E_{p(\mathbf z\mid\mathbf x,\Theta^{\text{old}})}
        \bigl[\log p(\mathbf X,\mathbf Z\!\mid\!\Theta)\bigr]
        =\sum_{\mathbf z} p(\mathbf Z\!\mid\!\mathbf X,\Theta^{\text{old}})
          \log p(\mathbf X,\mathbf Z\!\mid\!\Theta). \)\\
\textbf{M (Maximization) step}:\\
\quad Maximize \( Q \) with respect to \( \Theta \).\\
\quad For maximum-likelihood estimation (MLE):\;
\( \Theta^{\text{new}}=\arg\max_{\Theta}\, Q(\Theta,\Theta^{\text{old}}). \)\\
\quad For maximum-a-posteriori (MAP) estimation:\;
\( \Theta^{\text{new}}=\arg\max_{\Theta}\!\bigl\{ Q(\Theta,\Theta^{\text{old}})+\log p(\Theta) \bigr\}. \)\\
If the log-likelihood or the parameter values have not converged, set \( \Theta^{\text{old}}=\Theta^{\text{new}} \) and return to the E step.\\
The algorithm converges to a local maximum of \( p(\mathbf X\!\mid\!\Theta) \).


\textcolor{myyellow}{EM for Gaussian Mixture Model (GMM)}:\\
Initialize parameters \( \Theta=\{\pi_k,\mu_k,\Sigma_k\}_{k=1}^K \) 

Iterate until \( \log p(\mathbf X\!\mid\!\Theta) \) convergence :\\
\quad\textbf{E-step:}\\
\quad\(\displaystyle 
\gamma_{ik}=p(z_i=k\mid x_i,\Theta^{\text{old}})
          =\frac{\pi_k^{\text{old}}\,
                 \mathcal N(x_i\mid\mu_k^{\text{old}},\Sigma_k^{\text{old}})}
                 {\sum_{j=1}^K\pi_j^{\text{old}}\,
                 \mathcal N(x_i\mid\mu_j^{\text{old}},\Sigma_j^{\text{old}})}.
\)
\quad Expected complete-data log-likelihood:\\
\quad\(\displaystyle 
Q(\Theta,\Theta^{\text{old}})
      =\mathbb E_{Z\mid X,\Theta^{\text{old}}}
       [\log p(\mathbf X,\mathbf Z\mid\Theta)]
      =\sum_{i=1}^{N}\sum_{k=1}^{K}
        \gamma_{ik}\bigl(\log\pi_k+\log\mathcal N(x_i\mid\mu_k,\Sigma_k)\bigr).
\)

\textbf{M-step:} maximize \( Q \) with respect to \( \Theta \):
\quad  (with \( \sum_k\pi_k=1 \)):\\
\quad\(\displaystyle 
\pi_k^{\text{new}}
      =\frac{1}{N}\sum_{i=1}^{N}\gamma_{ik}.
\)
\quad\(\displaystyle 
\mu_k^{\text{new}}
      =\frac{\sum_{i=1}^{N}\gamma_{ik}x_i}{\sum_{i=1}^{N}\gamma_{ik}}.
\)\\
\quad\(\displaystyle 
\Sigma_k^{\text{new}}
      =\frac{\sum_{i=1}^{N}
              \gamma_{ik}(x_i-\mu_k^{\text{new}})(x_i-\mu_k^{\text{new}})^{\!\top}}
             {\sum_{i=1}^{N}\gamma_{ik}}.
\)\\

 \( p(\mathbf X,\mathbf Z\mid\Theta)=
\prod_{i=1}^{N}\pi_{z_i}\,\mathcal N(x_i\mid\mu_{z_i},\Sigma_{z_i}) \).\\
\( \log p(\mathbf X,\mathbf Z\mid\Theta)=
\sum_{i=1}^{N}\bigl(\log\pi_{z_i}+\log\mathcal N(x_i\mid\mu_{z_i},\Sigma_{z_i})\bigr) \).\\
Gaussian PDF in \( D \) dimensions: \( \mathcal N(x\mid\mu_k,\Sigma_k)=
(2\pi)^{-D/2}|\Sigma_k|^{-1/2}
\exp\!\bigl(-\tfrac12(x-\mu_k)^{\!\top}\Sigma_k^{-1}(x-\mu_k)\bigr) \).




\end{minipage}
%------------ Four column end -------------

\newpage

\begin{minipage}[t]{0.24\textwidth}
\textcolor{myred}{L23 PCA}:
\textcolor{myyellow}{Data Centering}:
Mean vector \( \mu=\dfrac{1}{N}\sum_{n=1}^{N}\mathbf x^{(n)} \).\\
Centered samples \( \tilde{\mathbf x}^{(n)}=\mathbf x^{(n)}-\mu\ \forall\,n \).\\
Stacked matrix \( X=\begin{bmatrix}\tilde{\mathbf x}^{(1)^{\!T}} \tilde{\mathbf x}^{(2)^{\!T}} \cdots \tilde{\mathbf x}^{(N)^{\!T}}\end{bmatrix}^T\in\mathbb R^{N\times D} \).\\
\textcolor{myyellow}{Sample Variance \& Covariance}:\\
One-dimensional variance \( \hat\sigma^{2}=\dfrac{1}{N}\sum_{i=1}^{N}\bigl(x^{(i)}-\hat\mu\bigr)^{2} \).\\
\( \Sigma_{jk}=\dfrac{1}{N}\sum_{i=1}^{N}\bigl(x_{j}^{(i)}-\hat\mu_{j}\bigr)\bigl(x_{k}^{(i)}-\hat\mu_{k}\bigr) \).\\
Matrix form \( \Sigma=\dfrac{1}{N}X^{T}X \).\\
\textcolor{myyellow}{Projection \& Reconstruction Error}:\\
Projection of \( \tilde{\mathbf x}^{(n)} \) onto unit vector \( \mathbf v \):
\( z^{(n)}=\mathbf v^{T}\tilde{\mathbf x}^{(n)} \).\\
Reconstruction error summed over data:\\  
\( 
\sum_{n=1}^{N}\bigl\|\tilde{\mathbf x}^{(n)}-(\mathbf v^{T}\tilde{\mathbf x}^{(n)})\mathbf v\bigr\|_{2}^{2}
       =\sum_{n=1}^{N}\bigl\|\tilde{\mathbf x}^{(n)}\bigr\|_{2}^{2}-\sum_{n=1}^{N}\bigl(\mathbf v^{T}\tilde{\mathbf x}^{(n)}\bigr)^{2}.
\)\\
Minimizing the error \(\iff\) maximizing the second term (projected variance).\\
\textcolor{myyellow}{Variance Maximization Form}:\\
\( 
\hat{\mathbf v}=\arg\max_{\mathbf v:\|\mathbf v\|_{2}^{2}=1}
        \sum_{n=1}^{N}\bigl(\mathbf v^{T}\tilde{\mathbf x}^{(n)}\bigr)^{2}
      =\arg\max_{\mathbf v:\|\mathbf v\|_{2}^{2}=1}
        \mathbf v^{T}\!\Bigl(\sum_{n=1}^{N}\tilde{\mathbf x}^{(n)}\tilde{\mathbf x}^{(n)^{\!T}}\Bigr)\mathbf v
      =\arg\max_{\mathbf v:\|\mathbf v\|_{2}^{2}=1}
        \mathbf v^{T}(X^{T}X)\mathbf v.
\)\\
\textcolor{myyellow}{Eigenvalue Problem}:\\
Lagrangian \( \mathcal L(\mathbf v,\lambda)=\mathbf v^{T}(X^{T}X)\mathbf v-\lambda(\|\mathbf v\|_{2}^{2}-1) \).\\
Gradient condition \( 2(X^{T}X)\hat{\mathbf v}-2\lambda\hat{\mathbf v}=0 \Rightarrow (X^{T}X)\hat{\mathbf v}=\lambda\hat{\mathbf v}. \)\\
Thus \( \hat{\mathbf v} \) is an eigenvector of \( X^{T}X \) (or \( \Sigma \)) and  
\( \lambda \) is the associated eigenvalue (variance along that component).\\
Principal components are the eigenvectors ordered by decreasing \( \lambda \);  
\( \lambda_i \) quantifies variance captured by component \( \hat{\mathbf v}_i \).\\
\textcolor{myyellow}{PCA Algorithm}:\\
1. Center data \( \mathbf X\rightarrow X \).\\
2. Compute covariance matrix \( \Sigma=\dfrac{1}{N}X^{T}X \).\\
3. Eigendecompose \( \Sigma\Rightarrow\{(\hat{\mathbf v}_i,\lambda_i)\} \).\\
4. Select top \( K \) eigenvectors \( V_{K}=[\hat{\mathbf v}_1,\dots,\hat{\mathbf v}_K] \).\\
5. Low-dim projection \( \mathbf z^{(n)}=V_{K}^{T}\tilde{\mathbf x}^{(n)} \).\\
Applications: visualization, noise reduction, improved generalization in downstream tasks.\\

\end{minipage}
\hfill
\begin{minipage}[t]{0.24\textwidth}
% ---------- SVM: Geometry ➜ Primal ➜ Dual ➜ Soft Margin ➜ Kernel ----------



\( \displaystyle
\min_{w,b}\;\tfrac12\lVert w\rVert^{2}\;
\text{s.t. }y_i\,(w^{\!\top}x_i+b)\ge1
\).

\textcolor{myyellow}{Lagrange Dual (Support Vectors emerge)}:\\
Lagrangian\\
\( \mathcal L(w,b,\alpha)
   =\tfrac12\lVert w\rVert^{2}
    -\!\sum_{i=1}^{m}\alpha_i\bigl[y_i\,(w^{\!\top}x_i+b)-1\bigr],\;
    \alpha_i\ge0. \)\\
Stationarity\\
\( w=\sum_{i}\alpha_i y_i x_i,\;
   \sum_{i}\alpha_i y_i=0. \)\\
Dual\\
\( \displaystyle
\max_{\alpha}\sum_{i}\alpha_i
       -\frac12\sum_{i,j}\alpha_i\alpha_j y_i y_j\,x_i^{\!\top}x_j,\;
   \alpha_i\ge0,\;
   \sum_i\alpha_i y_i=0.
\)\\
KKT\\
\( \alpha_i\bigl[y_i(w^{\!\top}x_i+b)-1\bigr]=0 \).
Only \( \alpha_i>0 \) are \emph{support vectors}.\\
\textcolor{myyellow}{Soft-Margin SVM (Hinge Loss)}:\\
Primal
\( \displaystyle
\min_{w,b,\xi}\tfrac12\lVert w\rVert^{2}
     +C\sum_{i=1}^{m}\xi_i,\;
y_i\,(w^{\!\top}x_i+b)\ge1-\xi_i,\;
\xi_i\ge0.
\)\\
Hinge loss \( \ell=\max\{0,1-y_i(w^{\!\top}x_i+b)\} \).\\
Dual\\
\( \displaystyle
\max_{\alpha}\sum_{i}\alpha_i
       -\tfrac12\sum_{i,j}\alpha_i\alpha_j y_i y_j\,x_i^{\!\top}x_j,\;
0\le\alpha_i\le C,\;
\sum_i\alpha_i y_i=0.
\)\\
SV classes: \( 0<\alpha_i<C \) (on/inside margin), \( \alpha_i=C \) (errors).

\textcolor{myyellow}{Kernel Trick – Non-linear SVM}:\\
Replace inner product: \( x_i^{\!\top}x_j \rightarrow K(x_i,x_j) \).\\
Dual\\
\( \displaystyle
\max_{\alpha}\sum_{i}\alpha_i
   -\tfrac12\sum_{i,j}\alpha_i\alpha_j y_i y_j\,K(x_i,x_j),\;
0\le\alpha_i\le C,\;
\sum_i\alpha_i y_i=0.
\)\\
Classifier
\( f(x)=\operatorname{sign}\!\bigl(\sum_{i}\alpha_i y_i K(x_i,x)+b\bigr) \).
\end{minipage}
\hfill
\begin{minipage}[t]{0.24\textwidth}

$ $


\end{minipage}
\hfill
\begin{minipage}[t]{0.24\textwidth}

$ $


\end{minipage}
\hfill
\end{document}
